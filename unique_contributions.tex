\documentclass[11pt,letterpaper]{article}
\usepackage[margin=1in]{geometry}
\usepackage[]{natbib}
\bibpunct[; ]{[}{]}{,}{n}{}{;} 
\bibliographystyle{unsrtnat}
\setcitestyle{plain}
\title{Unique Contributions}
\date{}

\begin{document}
\maketitle

\noindent High-throughput sequencing has accelerated the pace of discovery in all fields of genomics.
Agricultural plants provide a particularly excellent system for basic discovery, with their rich history of basic genetic research,  ample modern genetic resources, and the opportunity to sample or  replicate genotypes across a wide array of environments.
The country's most economically valuable crop, maize, is also one of the oldest model organisms, famous for its phenotypic and genetic diversity.
This diversity has led to a number of scientific discoveries, including heterosis \citep{shull1908composition}, recombination and crossing over \citep{creighton1931correlation}, meiotic drive \citep{rhoades1942preferential}, transposable elements \citep{mcclintock1950origin}, and the first cloned QTL in a plant \citep{doebley1997evolution}. 
Though other groups are using high-throughput sequencing to accelerate applied breeding, my group is uniquely poised to  exploit maize diversity using population genetic approaches to understand fundamental processes of evolutionary change in plant genomes.
%The data alone open a host of new possibilities: from identifying individual \emph{de novo} transposable element insertions to studying structural rearrangements to rare single-nucleotide mutations, as well as the ability to assess assessing the impact of these variants on expression, methylation, or chromatin state.
Familiar basic questions --- such as the origin and nature of adaptive mutations or whether most adaptive evolution is  geographically local ---  can now be re-addressed  genome-wide across a number of populations.
But genomic data also enable us to investigate processes that were previously intractable, such as the mechanisms that govern transposable element transposition and spread, how demographic change shapes the fitness consequences of new mutations, how selection drives genome size variation across environments, or how individual deleterious variants may combine to explain heterosis.
Going beyond crop improvement, pursuing these questions in the wild ancestral populations of maize will also advance our understanding of evolution and conservation in natural systems.


%Within the context of your field, provide a statement (not to exceed 250 words) of the contribution you are uniquely poised to make to your research field, and, where appropriate, potential to result in new approaches to human health problems that disproportionately affect individuals living in low resource settings.

\bibliography{jri.bib}
\end{document}
