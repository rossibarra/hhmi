\documentclass[11pt,letterpaper]{article}
\usepackage[margin=1in]{geometry}
\usepackage[]{natbib}
\bibpunct[; ]{[}{]}{,}{n}{}{;} 
\bibliographystyle{unsrtnat}
\setcitestyle{plain}
\title{Unique Contributions}
\date{}

\begin{document}
\maketitle

\noindent High throughput sequencing has dramatically accelerated the pace of discovery in all fields of genomics.
Agricultural plants provide a paritcularly excellent system for basic discovery, with their rich history of basic genetic research, ample modern genetic resources, and the opportunity to sample or even replicate genotypes across a wide array of environments.
The country's most economically valuable crop, maize, is also one of the oldest genetic model organisms, famous for its phenotypic and genetic diversity.
This diversity has directly led to an impressive number of scientific discoveries, including heterosis \citep{shull1908composition}, crossing over \citep{creighton1931correlation}, meitoic drive \citep{rhoades1942preferential}, transposable elements \citep{mcclintock1950origin}, and the first cloned QTL in any plant \citep{doebley1997evolution}. 
Though a number of groups are applying high-throughput sequencing to accelerate applied breeding, my group is uniquely poised to  exploit maize diversity to understand fundamental processes of evolutionary change in plant genomes.
The data alone open a host of new possibilities: from identifying individual \emph{de novo} transposable element insertions to studying structural rearrangements to rare single-nucleotide mutations, and in each case assessing the impact of these variants on expression, methylation, or chromatin state.
But the application of population genetic theory also lets us interrogate processes, including basic questions such as what proportion of adaptive change is geographically local or what is the role of standing variation. 
We can now begin to explicitly test complex hypotheses including how demography shapes the fitness consequences of new mutations, how individual deleterious variants may combine to provide hybrid vigor, or how the balance of selection and transposition may shape genome size variation.


%Within the context of your field, provide a statement (not to exceed 250 words) of the contribution you are uniquely poised to make to your research field, and, where appropriate, potential to result in new approaches to human health problems that disproportionately affect individuals living in low resource settings.

\bibliography{jri.bib}
\end{document}
