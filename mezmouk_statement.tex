\documentclass[11pt,letterpaper]{article}
\usepackage[margin=1in]{geometry}
\usepackage[]{natbib}
\bibpunct[; ]{[}{]}{,}{n}{}{;} 
\bibliographystyle{unsrtnat}
\setcitestyle{plain}
\usepackage{url}
\renewcommand{\UrlFont}{\small\tt}
\begin{document}

\large{\noindent {\bf The pattern and distribution of deleterious mutations in maize.} 
Sofiane Mezmouk and Jeffrey Ross-Ibarra.
\textsc{G3: Genes— Genomes— Genetics}, 4(1):163–171, 2014.}

\vspace{1cm}

\noindent The observation of hybrid vigor, or heterosis, in crosses between inbred lines was first observed more than a century ago \citep{shull1908composition,shull1914duplicate}, but the molecular mechanisms underlying this observation remain elusive to this day \citep{schnable2013progress, chen2013genomic}.
Nonetheless, theory has long suggested that much of heterosis can be explained in quantitative genetics terms as deleterious variants that are at least partially recessive \citep{charlesworth2009genetics}.
This paper presented the first evidence for a widespread role of deleterious variants in patterning heterosis and phenotypic variation in maize \citep{mezmouk2014pattern}.
We used genome-wide genotyping and biochemical models of amino-acid effects to predict the functional consequences of variation across a large panel of maize inbred lines.
Our population genetic analysis revealed that most deleterious variants are rare and under relatively weak selection.
Applying genome-wide-association mapping approaches, we showed that genes associated with heterosis are enriched for deleterious variants for every phenotype evaluated, and that phenotypic variation among inbreds was in many cases also associated with deleterious variants.\\

\noindent These results have a number of implications for understanding the genetic basis of heterosis and phenotypic variation in crop plants. 
They explain the steady improvement in yield of inbred lines --- an observation that would not be expected if most variants were strongly deleterious and purged quickly from populations.
Our results build on our previous observations of selection to maintain heterozygosity in low recombination regions \citep{gore2009a-first-generation} to suggest that much of heterosis in  maize can be explained by numerous, linked, weakly deleterious variants in large pericentromeric blocks; this explanation is borne out by our ongoing analysis of experimental populations \citep{gerke2013genomic}.
Finally, these results predict that, because they are under relatively weak selection, the abundance of deleterious variants in populations may be driven not by breeders selecting for increased yield but instead by demographic change and the effects of selection on linked sites.


\bibliography{jri.bib}
\end{document}
