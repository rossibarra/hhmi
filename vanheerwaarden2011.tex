\documentclass[11pt,letterpaper]{article}
\usepackage[margin=1in]{geometry}
\usepackage{color}
\usepackage{graphicx}
\usepackage{url}
\usepackage{multicol}
\usepackage{wrapfig}
\usepackage{amsmath}
\usepackage{amssymb}
\usepackage{caption}
\usepackage{subcaption}
\usepackage[]{natbib}
\bibpunct[; ]{[}{]}{,}{n}{}{;} 
\bibliographystyle{unsrtnat}
\setcitestyle{plain}
\usepackage[usenames,dvipsnames,svgnames,table]{xcolor}
\usepackage[innercaption]{sidecap} % side captions
\sidecaptionvpos{figure}{c}
\begin{document}

\large{\noindent {\bf Genetic signals of origin, spread, and introgression in a large sample of maize landraces.} 
J. van Heerwaarden, J. Doebley, W. H. Briggs, J. C. Glaubitz, M. M. Goodman, J. J. S. Gonzalez, and J. Ross-Ibarra. 
\textsc{Proceedings of the National Academy of Sciences}, 108(3):1088–1092, 2011.}

\vspace{1cm}

\noindent Throughout most of the 20th century, the origins of domesticated maize remained the subject of considerable controversy \citep{doebley2004genetics}. 
The debate was finally resolved in 2002, with genetic analysis from John Doebley's group \citep{matsuoka2002single} confirming the earlier crossing experiments by George Beadle \citep{beadle1972mystery} and identifying the teosinte \emph{Zea mays.} ssp. \emph{parvilumis} as the direct progenitor of maize. 
This genetic data, however, presented a new paradox: the extant maize populations most genetically similar to ssp. \emph{parvilumis} were from highlands of central Mexico, outside the the historical range of ssp. \emph{parvilumis} \citep{hufford2012inferences} and far from the archaeological sites showing earliest evidence of maize cultivation \citep{hastorf2009rio, piperno2009starch, pohl2007microfossil}.
In this paper we used a sample of more than 1200 maize lines from across the Americas to resolve this paradox and show that the genetic data are in fact consistent with archaeological and ecological evidence.
Our analysis revealed that the observation of genetic similarity in highland maize was due to admixture with another wild relative of maize, ssp. {mexicana} \citep{van2011genetic}

\bibliography{jri.bib}
\end{document}
