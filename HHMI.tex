%Project Narrative
\documentclass[12pt]{article}
\usepackage[margin=1.225in]{geometry}
\usepackage{color}
\usepackage{graphicx}
\usepackage{url}
\usepackage{multicol}
\usepackage{wrapfig}
\usepackage{amsmath}
\usepackage{amssymb}
\usepackage{caption}
\usepackage{subcaption}
\usepackage[round]{natbib}
\bibliographystyle{abbrvnat}
\setcitestyle{authoryear,open={(},close={)}}
\usepackage[usenames,dvipsnames,svgnames,table]{xcolor}
\usepackage[innercaption]{sidecap} % side captions
\sidecaptionvpos{figure}{c}

\begin{document}

\title{\vspace{-5ex}Maize as a model of the impact of demography and selection in complex genomes\vspace{-4ex}}
\author{}
\date{}
\maketitle

\section*{Maize as a model for complex genomes}

Genome size variation in plants, humans. 
Differences in large complex genomes -- Fraser \citep{fraser2013gene} vs. \citep{pyhajarvi2013complex} or \citep{hancock2011adaptation}

\section*{Notes}

\citep{peischl2015expansion} predict more u-shaped SFS in expanded pops, also more homozygous deleterious. this is seen in humans. Li's results in maize




%
%\begin{SCfigure}
%\includegraphics[width=0.45\linewidth]{joost_diversity.png}
%\caption{Changes in genetic diversity (represented as the effective number of ancestors) of the three primary maize heterotic groups (SS: stiff-stalk; NSS: non-stiff-stalk; IDT: iodent). Inbreds are divided into eras representing different time periods: 1:1930-1950; 2:1950-1980; 3:1985-1992. Figure from \citet{van2012historical}.} 
%\label{fig:diversity}
%\end{SCfigure}
%


%\begin{figure}
%\centering
%\includegraphics[width=0.7\linewidth]{yield.png}
%\caption{Blah} 
%\label{fig:piecewise}
%\end{figure}



\newpage
\bibliography{jri.bib}
\end{document}
