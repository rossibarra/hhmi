\documentclass[11pt,letterpaper]{article}
\usepackage[margin=1in]{geometry}
\usepackage[]{natbib}
\bibpunct[; ]{[}{]}{,}{n}{}{;} 
\bibliographystyle{unsrtnat}
\setcitestyle{plain}

\title{Significant Research Achievements}
\date{}

\begin{document}
\maketitle
\noindent

\noindent My research has applied population genetics to  natural and domesticated populations of the genus \emph{Zea} to understand the genomic basis of evolutionary change.

\subsection*{Revising models of crop domestication}
Traditional mapping approaches argued that a handful of loci could effect much of the evolutionary change during domestication, and classical models of selective sweeps suggested such a pattern should result from strong selection on new mutations. 
Our population genomic analysis of maize and teosinte genomes \citep{hufford2012comparative} identified hundreds of loci targeted by selection during maize domestication and suggested that selection on new mutations is quite rare.
In fact, our detailed looks at individual loci of large effect cloned by traditional mapping methods show that even at these loci selection evolution has occurred via selection on multiple beneficial mutations \citep{wills2013many} or standing genetic variation \citep{wills2013many, studer2011identification}.

\subsection*{Identifying new sources of adaptive variation}
Protein-coding genes have often been considered the major source of functional variation, and evidence from small model genomes such as \emph{Arabidopsis} \citep[e.g.][]{hancock2011adaptation} supports such claims. 
However, most flowering plants have large genomes, like maize, and our work has indicated that in such large genomes other forms of variation likely play a much more important role.  
We have shown that selection has taken extensive advantage of regulatory \citep{swanson2012reshaping,pyhajarvi2013complex} and structural variation --- including inversions \citep{pyhajarvi2013complex,fang2012megabase}, transposable element insertions \citep{studer2011identification,makarevitch2015transposable}, and likely copy-number variation \citep{chia2012maize} --- in both natural and domesticated populations. We believe such noncoding variation will be shown responsible for a substantial fraction of adaptive variation in most plant genomes.


%\subsection*{Gene flow as an important evolutionary force}
%Gene flow among populations can obscure genetic relationships and 
%We documented the adaptive potential of gene flow from crop wild relatives  \citep{hufford2013genomic,Takuno15062015}, underscoring the importance of such biodiversity for conservation and future breeding.
%
%%A 250-word summary of your most significant research achievements.


\bibliography{jri.bib}
\end{document}
