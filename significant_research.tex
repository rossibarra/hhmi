\documentclass[11pt,letterpaper]{article}
\usepackage[margin=1in]{geometry}
\usepackage[]{natbib}
\bibpunct[; ]{[}{]}{,}{n}{}{;} 
\bibliographystyle{unsrtnat}
\setcitestyle{plain}

\title{Significant Research Achievements}
\date{}

\begin{document}
\maketitle
\noindent

\noindent My research has applied population genetics to  natural and domesticated populations of \emph{Zea} to udnerstand the genomic basis of evolutionary change.

\subsection*{Revising evolutionary models of crop domestication}
Decades of traditional mapping approaches argued that a handful of loci could effect much of the evolutionary change during domestication, and classical models of selective sweeps  suggested that such loci should result from strong selection on new mutations. 
Our population genomic analysis of maize and teosinte genomes  \citep{hufford2012comparative} identified hundreds of loci targeted by selection during maize domestication, and suggested that selection on new mutations is likely quite rare.
In fact, our detailed looks at individual loci of large effect cloned by traditional mapping methods show that even at these loci selection evolution has occurred via selection on standing genetic variation on multiple beneficial mutations \citep{wills2013many} or standing genetic variation \citep{wills2013many, studer2011identification}.

\subsection*{Unique sources of adaptive variation in complex genomes}
Variation in protein-coding genes is often considered the major source of functional variation, and evidence from small genomes such as \emph{Arabidopsis} \citep[e.g.][]{hancock2011adaptation} supports such claims. 
The maize genome, more than 2Gb in size, is far closer to the average flowering plant genome and thus likely more representative than model systems chosen for their small genomes. 
We have shown that selection in the genus \emph{Zea} has taken extensive advantage of regulatory \citep{swanson2012reshaping,pyhajarvi2013complex} and structural variation --- including inversions\citep{pyhajarvi2013complex,fang2012megabase}, transposable element insertions \citep{studer2011identification,makarevitch2015transposable}, and likely copy-number variation \citep{chia2012maize}	--- in both natural and domesticated populations. 

\subsection*{Gene flow as an important evolutionary force}
Gene flow among populations can obscure genetic relationships and 
We documented the adaptive potential of gene flow from crop wild relatives  \citep{hufford2013genomic,Takuno15062015}, underscoring the importance of such biodiversity for conservation and future breeding.

%A 250-word summary of your most significant research achievements.


\bibliography{jri.bib}
\end{document}
