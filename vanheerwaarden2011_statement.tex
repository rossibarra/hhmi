\documentclass[11pt,letterpaper]{article}
\usepackage[margin=1in]{geometry}
\usepackage[]{natbib}
\bibpunct[; ]{[}{]}{,}{n}{}{;} 
\bibliographystyle{unsrtnat}
\setcitestyle{plain}
\begin{document}

\large{\noindent {\bf Genetic signals of origin, spread, and introgression in a large sample of maize landraces.} 
J. van Heerwaarden, J. Doebley, W. H. Briggs, J. C. Glaubitz, M. M. Goodman, J. J. S. Gonz\'{a}lez, and J. Ross-Ibarra. 
\textsc{Proceedings of the National Academy of Sciences}, 108(3):1088-1092, 2011.}

\vspace{1cm}

\noindent Throughout most of the 20th century, the genetic and geographic origins of domesticated maize remained the subject of considerable controversy \citep{doebley2004genetics}. 
The identity of the wild ancestor was finally resolved in 2002, with genetic analysis \citep{matsuoka2002single} confirming earlier crossing experiments \citep{beadle1972mystery} identifying the teosinte \emph{Zea mays.} ssp. \emph{parviglumis} as the direct progenitor of maize. 
These results, however, were inconsistent with geography: the extant maize populations most genetically similar to ssp. \emph{parviglumis} were found in the highlands of central Mexico, outside the historical range of ssp. \emph{parviglumis} \citep{hufford2012inferences} and far from archaeological sites showing the earliest evidence of maize cultivation \citep{hastorf2009rio, piperno2009starch, pohl2007microfossil}.
In this paper, my group successfully resolved this paradox \citep{van2011genetic}, effectively reconciling the genetic, archaeological and ecological evidence.
We genotyped more than 1200 maize lines from across the Americas and showed that the genetic similarity of highland maize to ssp. \emph{parviglumis} was due not to recent ancestry but instead admixture with another wild relative of maize, ssp. \emph{mexicana}.
We then applied population genetic approaches to reconstruct ancestral maize allele frequencies and assess differentiation from ancestral frequencies across the geographic range of maize in the Americas.
We found that extant maize populations most similar to ancestral frequencies were precisely those from lower elevations, well within the range of the wild ancestor \emph{parviglumis} and close to the oldest archaeological sites.\\

\noindent After more than a century of debate, this work represents what is essentially the final chapter on the origins of domesticated maize, consolidating earlier genetic and archaeological data into a single cohesive picture of the origin of one of the world's most important cultivated specices.
Since publication, this paper has been cited more than 100 times and is now a standard reference in discussions of domestication and the impact of introgression on reconstructing geographic origins.

\bibliography{jri.bib}
\end{document}
